%%%%%%%%%%%%%%%%%%%%%%%%%%%%%%%%%%%%%%%%%%%%%%%%%%
%
% Chapter:  Leadership & Management
%
%%%%%%%%%%%%%%%%%%%%%%%%%%%%%%%%%%%%%%%%%%%%%%%%%%

\chapter{Leadership \& Management}

\begin{quote}
``Don't worry about people stealing your ideas. If your ideas are that good, you'll have to ram them down people's throats.'' -- Howard Aiken
\end{quote}

TODO: Tie back to Howard Aiken's quote somewhere later in the chapter (end of the leadership section?).


\section{The Face of Leadership}

It seems fair to say that most people who call themselves leaders or managers could not define what either term means.

The Oxford English Dictionary unhelpfully defines leadership as ``the act of leading a group of people or an organization'', and a leader as, ``the person who leads or commands a group, organization, or country.'' We obviously need to do better than looking to the canonical definition of our language.

Leadership is rightly approached with fear and trepidation, except by those few megalomaniacs who should not be given the responsibility. Management in its turn is often viewed as a ``black art'' by people expected to practice it, encompassing as it does a situationally dependent mix of tools, techniques, and applied psychology. It is difficult to find authoritative references. To anyone with a requirement to shepherd an organization however, it is most critical to define the differences and discover a personal path forward.

There are no lack of books purporting to demystify either leadership or management.  They abound.  The problem is that pop psychology is not the best guide to dealing with the complexities of real human societies.  Real human societies are the most complex thing ever built by people. Consider how contrary to human emotional tendencies are the workings of a democratic legislature and I think you will see what I mean.

It is often said that leaders are born, not made.  In contrast it is assumed that managers can be made.  There is no doubt that being charismatic helps a leader and that the procedures of management may be learned.  However, I think that one naturally goes hand-in-hand with the other and that a lack in one will illuminate weaknesses in the other.  It is my contention that both are learned skills and may only be assisted or hindered by one's natural gifts.

It is hard to find good books yet I think one can learn something from them.  I recommend reading a few biographies of leaders judged by history to have been superior.  George Patton\index{Patton, General George S.}'s ``War as I Knew it'' is a good start, as is Lee Iacocca\index{Iacocca, Lido Anthony ``Lee''}'s recollections of his experiences in the automobile industry. The collected letters of John and Abigail Adams\index{Adams, President John and Abigail} of US revolutionary fame round out a first collection. The most interesting collection of software-specific vignettes I know is \underline{Open Sources: Voices from the Open Source Revolution}\footnote{Available entirely online at \href{http://www.oreilly.com/openbook/opensources/book/index.html}{http://www.oreilly.com/openbook/opensources/book/index.html}}. You will assuredly want to ignore Paul Vixie's dated and facile description of software engineering in the 1990s, and instead focus on Richard Stallman's and Robert Young's descriptions of decision-making at the cusp of an era.

Leadership and management are often confused by those who would benefit most by clearly separating them.  Militaries, and often politicians, traditionally refer to all forms of oversight and control of people as ``leadership'' while the term ``management'' is more commonly used in business.  I make a finer distinction:  Leadership is the art of motivating people to operate as a team and management is the science of making those efforts effective.  Leadership is encouragement, coaching and psychology.  Management is your toolbox of procedures, time sheets, charts and graphs.  Put another way, leadership is about people and management is about organizations.

Leadership has one other critical aspect:  Accountability. The buck stops with the leader of an organization and that person must be both willing and able to make difficult decisions in a timely matter.  Failure to do so can lead to the failure of projects and even collapse of organizations.

\vspace{20pt}
\setlength{\fboxsep}{10pt}%
\shadowbox{
\begin{minipage}{11cm}
       \begin{center}
Leadership is the art of creating a team.\\
Management is the science of making those efforts effective.
       \end{center}
       \end{minipage}
}

The management consultant Peter F. Drucker\index{Drucker, Peter F.} had something very similar in mind when he said, ``Management is doing things right; leadership is doing the right things.''

Where Drucker and I differ is this: My definition of leadership stops when a team is formed. There is no mention of a \textit{mission}, nor where a leader is expected to lead. Who should say that it is the leader who chooses? This is very often not so. The mission may come from outside the team, as is commonplace in corporate software development, or even from the team members themselves. The latter is common in Open Source software projects. It is the mission that determines what Drucker's ``right things'' are.

The American leadership theorist James M. Burns\index{Burns, James M.} has perhaps the most influence on the academic debate when he separated ``transactional'' leadership from ``transformational'' leadership\cite{Burns-1978}. It is generally the former that I refer to as management, and the latter as leadership.

Software team members will generally express their dissatisfaction if the mission isn't worth the effort asked, or if it is simply wrongheaded. That is, no doubt, due to the generally higher degree of intelligence possessed of people who choose of their own free will to wrestle with machinery. The human species would have had a much more quiet history if all groups of people expressed dissatisfaction so readily, and so clearly.

A good leader will listen to the tenor of their group's conversations to constantly determine the level of satisfaction, and adjust their actions accordingly. Is this, then, not ``leading from behind''? If one is merely responding to the group's needs, in what sense can a leader be setting direction? Recall that maximizing a group's happiness is only one portion of a leader's responsibilities. The other, making the team productive, is equally important and is where the tools of management can help.


\section{A Word of Caution}

It has become your job to make a team from a group of disparate individuals. What, you might rightly ask, makes a team? The answer is surprisingly simple. Teams are made by establishing a shared identity.

Shared identity is a relatively straightforward concept to create. There are groups of people who get together because they share a common interest, such as dog breeding, or acting, or reading mysteries. There are groups based around religion, or ethnic identity, or even because they happen to share a small town. Software teams come nearly ready made in this sense, especially when they share the development and maintenance of a software product. A good leader can use the identity related to a product much more effectively than less meaningful identities, such as merely being an employee of a company.

However, and here is the warning, shared identity can also be created quickly and easily by convincing a group of people that they possess some form of special knowledge. That special knowledge separates them from everyone else. One often sees this in software development around conceptually difficult specialities, such as functional or logic programming, or in the mastery of a powerful yet esoteric programming language such as Smalltalk or Haskell.

Basing team identity on special knowledge is so often done because the idea is seductive. The temptation to create a bonded group quickly, and to harness the power that comes with it, is an easy way out. There is a dark side to such a decision. Basing a team on special knowledge, in the presence of a strong leader, is what defines a cult. Cultish behavior can and most often does result.

Taken to an extreme, cultish behavior leads to the excesses of a Stalin, or a Mao, or a Hitler, as well as to the formation of religious cults. On a smaller and more everyday scale, it can lead to a team that "believes their own press", and stops seeing the many clear indications that their products have critical flaws.

Cultish behavior in a business setting is how we get to the ``normalized deviancy'' discovered by Enron, and Volkswagen\footnote{For a particularly succinct description of Volkswagen's 2015 emissions scandal, see the  magazine article ``What Was Volkswagen Thinking?'' by Jerry Useem in the January-February 2016 issue of The Atlantic, \hyperlink{http://www.theatlantic.com/magazine/archive/2016/01/what-was-volkswagen-thinking/419127/}{http://www.theatlantic.com/magazine/archive/2016/01/what-was-volkswagen-thinking/419127/}.}.

The cure for cultish behavior and ``normalized deviancy'' is for both leaders and their teammates to constantly check themselves against outside opinion. If your actions seem odd to a domestic partner, or a close friend, they might very well be. Checking yourself and your team for the presence of cultish behavior is another in the long line of balancing acts. 

Leaders are not always thanked for doing the right thing, which makes leadership a lonely art. Many people do not wish to be told the objective truth, and would rather yield to the deep emotional comfort of a strong group identity. This is why we have strong cultural warnings about ``shooting the messenger", and why the second century CE Stoic philosopher (and Roman emperor) Marcus Aurelius warned us that a leader's job is, ``to do good and be damned.''\footnote{Marcus Aurelius, Meditations, book 7, section 36.}


\section{The Many Faces of Management}

According to my definition, there are many forms of management, but only one of leadership.  One may practice project management, personnel management, logistics management, etc.

One should not be insulted by being called a manager, nor feel unreasonably buoyed by being called a leader. The two are opposite sides of a coin, and cannot be separated.

Not all of the lessons of management are applicable to the software industry. Having a conception of the many faces of management can, however, be beneficial to the software team leader. It helps to place your own role in the greater schemes of human endeavor, and can give you ways to conceptualize the problems you will invariably encounter.


\subsection{Managerial Concepts}

Managers must understand their organization, and demonstrate that understanding.  We have all heard someone say, ``Everybody knows that it is wrong but management still makes us do it that way.''  This is a sign that those in charge don't know what is going on, don't listen to the people doing actual work, or (at best) haven't clearly communicated their reasoning to the rest of the organization.  All are critical errors in management.

Luckily, it is easy to correct such mistakes if one is willing to listen and learn.  The Total Quality Management (TQM) movement\footnote{TQM led directly to the adoption of the ISO 9000 series of management standards in 1987.} that began in the 1970s started with the simple realization that it is important to listen to assembly line workers' opinions.  They often discover ways to increase efficiency of operations. Unfortunately for those looking for a silver bullet, Total Quality ideas do not readily translate to other environments. They are particularly unsuited to non-procedural environments (such as research and development operations).

One of the best (and simple) ways to get to know an organization is still to talk to the people within it.  All of them. Regularly.  This is known as ``management by walking around'' (WBWA), also known by some as ``management by wandering around''. MBWA was a mainstay of the much-vaunted HP Way, used to great affect by Bill Hewlett and David Packard in their Silicon Valley company. MBWA is the best way, in my opinion, to understand small organizations, less than approximately 50 people - not by coincidence within the average size of a traditional human hunter-gatherer group.

Larger organizations are forced to implement progressively more hierarchy. Recent trends have been to flatten hierarchies as much as possible.  The degree of hierarchy is totally dependent on the goals of an organization.  Procedural organizations will generally require much more hierarchy than creative ones. Too much hierarchy decreases both creativity and productivity. Lack of necessary hierarchy overburdens management.

Managers of hierarchical organizations should still ``walk around'', if only to spot-check and get a feel for what is happening.  Hiding in an office will teach you exactly nothing about an organization.  Managers of hierarchies must meet regularly with those in charge of sections.  The amount of hierarchy must reflect the manager's ability to absorb section reports. In any case, organizational units should never exceed Dunbar's Number\footnote{British anthropologist Robin Dunbar theorized in 1992 that physical limitations of the human neocortex correlate to the number of stable social relationships that may be formed by an individual.} (approximately 150 individuals) without several hierarchical layers.

Managers should not be expected to know everything about their organizations, nor to be infallible!  In most organizations, a manager can afford the luxury of being a human being.  A possible exception exists for those in highly procedural environments where lives or property is at risk when an error occurs, such as a nuclear power station.  Even there, one must allow for human fallibility by building checks and balances into both social and technological systems.  The realization that humans routinely make mistakes is responsible for aircraft pilots having mandatory rest requirements, and for multiple nuclear weapons launch officers being required to agree before weapons activation.  In organizations where humans routinely operate while tired or under stress, procedural checks and balances must be followed.

Some managers follow a thumb rule of ``never asking a question for which you don't already know the answer".  This is a great idea if the manager is using the question as a training opportunity (to make the subject think about possible answers or to test the subject).  It is a horrible idea when the manager needs to determine the state of an organization and fails to ask critical questions.  In procedural environments, one might be expected to master a finite set of procedures over many years before being in charge.  It may be appropriate in those environments to ask only questions to which one already knows the answer.  Most organizations would be better managed if managers risked showing ignorance and asked legitimate questions.

Inexperienced managers often wish to appear more knowledgable than they are.  By failing to ask questions they ensure their continued ignorance and (in the end) fool nobody. The way to become an experienced manager is to learn your organization and the trade of management.  Ask questions. Respect will follow when you have earned it.

Good managers in all but the most mind-numbing of procedural organizations establish a culture of continuous learning. This can be in the form of training, encouragement of continued education, informal counseling, or some less formal form (such as simply reading books and leaving them around for others).

Training takes time.  It is also the best time that you can spend.  Constantly try to encourage people to the next level. Many managers ``build empires'' by protecting their knowledge. Instead, try training everyone to replace their superiors. You will be amazed at the results.  When you take the time to train people, they will notice that you care and their performance will rise.  Training is a very effective mechanism for team building, especially in organizations full of bright, capable people.

Software development organizations are some of the most fun to manage. They invariably consist of people more intelligent than the general population, and often have the advantage of a clearly defined shared goal. Software developers are renowned for their particular interests, hobbies, and personality types. Mastering the means of engaging with software developers as their own special category of people will pay off.


\subsection{Project Management}

Software team leaders may be wary of project management due to the explicit removal of the term ``project manager'' from the Scrum methodology. Project management was developed in a social framework of top-down control that seems to conflict directly with the team-oriented approaches of many software development frameworks. But the role played by a project manager is not entirely removed from those methodologies; they are present and simply distributed differently. It can therefore be useful to understand the term project management, and a bit of its history.

Project management is basically the application of applied economics to a project. Engineering as a discipline is also often described this way. In other words, almost anyone can get a job done given unlimited time and an unlimited budget.  Project management is a mechanism via which one ensures that a job may be successfully completed within a given budget.  Alternately, the tools of project management may be used to show that a job cannot be completed within the budget assigned.

Software teams generally struggle with the difficult task of estimating the amounts of time, effort, and budget are likely to be necessary to complete a project. There are good reasons for this, which are discussed in Chapter \ref{ch:SETheory}. Estimation, however difficult, remains a critical managerial task.

Project management can theoretically be as easy as choosing a methodology that suits your particular situation and learning how to apply it.  In mature disciplines, such as civil engineering, nobody discusses methodologies.  It would be foolish to ask a civil engineer which methodology they use to build a road.  There is only one, fine tuned over many decades to suit weather conditions, available road materials and other variables.  In immature disciplines, such as software engineering, methodologies abound. The number of competing methodologies is itself a sign of a rapidly changing industry.

The first software engineers borrowed the Waterfall methodology from older engineering fields. This is a wonderful way to track complex projects if the specification for the end product is not anticipated to change much during development. Stable requirements can, and do, occur in software projects, but it is an uncommon case. In response, software engineers developed new methodologies such as Rapid Application Development (RAD), the later Spiral RAD and, more recently, the so-called Agile methodologies such as Extreme Programming (XP) and Scrum.  Agile methodologies are appropriate for situations where it is difficult to determine a specification in advance (e.g. when no one group of people can adequately map available technology to known business problems), or when the specification is expected to change a great deal during development and maintenance.

Some of the great mistakes of project management are:
\begin{itemize}
\item Choosing the wrong methodology;
\item Assuming that a methodology will work for a given
        situation perfectly without modification;
\item Failing to adjust a methodology to local conditions;
\item Failing to adequately track a project's progress;
\item Over reliance on numbers or methodologies in place of
        an understanding of the project;
\item Failing to take adequate corrective action once a
        project is anticipated to be off schedule or budget.
\end{itemize}

All of these can be fatal errors and lead to a project's demise.  The last one may be viewed as a failure in leadership.

Traditional project managers master the use of conceptual tools to track a project's progress, such as Gantt and Pert charts.  These tools are generally taught in project management classes.  Gantt charts show task scheduling and Pert charts show task dependencies.  The two may be combined with task resource estimations to produce a critical path diagram which shows how fast a project may be completed in linear time.  All of these tools are critical to controlling a project managed with the Waterfall methodology.

In spite of their age, all project managers should learn and selectively apply these tools to their projects whenever requirements actually \textit{are} known up front, and expected to change slowly. Agile methodologies may eschew them, but it should be no surprise that modern Agile and Kanban tools, such as a Sprint Board, reintroduce (modified and time-scoped) tasking charts into software processes.

Some projects may have phases that are best handled by a combination of techniques. Knowing the available tools often keeps you from inventing your own. For example, you might be assigned to assist with the organization of an office move. Traditional project management tools and techniques then apply.

Another Waterfall concept is the formal specification. Traditional engineering teams produce functional (usage) and technical (implementation) specifications before beginning development.  Documentation is produced alongside development tasks as a parallel deliverable.  It may or may not be appropriate to develop specifications (or even documentation) for a software project, but any project manager worth the name should know how to produce them for those times when they are required.

The only thing that makes a ``good'' project manager is the ability to complete projects on time and on budget.  Under time and under budget is better.  Good project managers pride themselves on this ability. It is more challenging to do this for software projects than in any other field of engineering.

Good project managers posses one common trait:  They constantly think about their project.  Yes, this can interfere with social activities.  Good project managers worry, just a little bit, all the time.  This is because they are constantly aware of the state of their project and know what may go wrong.  They can calculate the impact of likely problems on schedule, cost and resourcing.  They prepare for the future before they future bites them (usually hard and in the middle of the night).  Good project managers thrive on the ``game'' of delivery.  Bad ones burn out.  Because of this worry, not everyone makes a good project manager.


\subsection{Logistics Management}

Plan ahead.  What will your team need next week, next month, and next year? Logistics management is an overly-fancy name for ensuring that your team has the materials they need to perform optimally.

Make certain that everyone has the tools and material necessary to do their jobs.  Plan to have a desk, chair, computer or other necessary equipment available on the \textit{first} day that a new employee joins. You never have a second opportunity to make a first impression.

The economic costs related to people, such as salaries, bonuses, holiday time, retirement savings, and other associated expenses, nearly always dominate the budget for a typical software organization. That means that spending should be aimed at keeping people productive.

\textbf{TODO}: Use an example budget
  for floor space, salaries, retirement, bonuses, equipment.
  Contrast with the budget for an aircraft carrier: even then
  people costs dominate.

A nice working environment will make people happy.  There is no reason to make people miserable.  Miserable people are not productive.  Slave labor is terribly inefficient, and centuries of brutal history have more than adequately demonstrated that it is simply a bad idea. Instead, relish your ability to create and maintain a positive and productive environment.

Failures of environment are the fault of the manager, even when resources are scare. Will you constantly complain that upper management does not give you the resources you want, or will you do well with what you have? Can you think of creative ways to better use what you can get? Software development organizations are generally better off economically than others. If you can't put emphasis on a good working environment, who can?

Thinking about logistics management can keep little things from getting in the way of productivity. This is especially important in software organizations, because programmers need time to just think. They should not be bothered by lack of toilet paper, running out of coffee or tea, or inexplicably losing access to basis services. Proper performance of logistics management can keep people from wasting valuable time, and from gossipy complaining -- that other bane of the happy workplace.


\section{Some Thoughts on Micromanagement}

Micromanagement is bound to be a temptation in environments where a (generally new) manager knows more about the way to do something than the person who should be doing the job. This is incredibly dangerous and is to be avoided at all costs.  No matter how tempting, \textbf{do not} micromanage.

If you really can do a job better \textit{and} the job is time critical \textit{and} there is no opportunity for training \textit{and} there is really no other choice after substantial thought, then simply assign the task to yourself.  Do not ``step in'' and perform a task while it was assigned to someone else. Doing so will seriously damage morale and is almost never worth the costs.  To make matters worse, you will be labelled a ``micromanager''. That is not a compliment.

A much better management style is to think ahead (as with all management tasks) and assign tasks to people based on their abilities.  Tend to assign tasks to people that are \textit{slightly} beyond their abilities, whenever possible.  When you encounter a situation which tempts you to micromanage, train the assigned person instead whenever possible, even if a schedule slips. Training always pays off in the long term.  The result will be high morale, a better team and a better organizational result.

Perhaps the final word should go to organizational consultant John Stocker, who sums up the dangers of micromanagement beautifully. Note how he equates micromanagement with other forms of bullying:

\begin{quotation}
``Authority -- when abused through micromanagement, intimidation, or verbal or nonverbal threats -- makes people shut down \& productivity ceases.''
\end{quotation}

There will be times to exert great control (e.g. over leave scheduling, budget, deliverables) and times to let the team rest.  Don't push hard all the time.  Be occasionally magnanimous.  Think hard every day about whether more or less control should be exerted today.  Once a team is established, back off and let them run with it.  If they do so, they are a team and you are a leader.


% If the chapter ends in an odd page, you may want to skip having the page
%  number in the empty page
\newpage
\thispagestyle{empty}



