%%%%%%%%%%%%%%%%%%%%%%%%%%%%%%%%%%%%%%%%%%%%%%%%%%
%
% Chapter:  Congratulations! Now What?
%
%%%%%%%%%%%%%%%%%%%%%%%%%%%%%%%%%%%%%%%%%%%%%%%%%%

\chapter{Congratulations! Now What?}

\begin{quote}
``Nothing in the world can take the place of persistence.  Talent will not; nothing is more common than unsuccessful men with talent. Genius will not; unrewarded genius is almost a proverb. Education will not; the world is full of educated derelicts. Persistence and determination alone are omnipotent.'' -- Calvin Coolidge
\end{quote}

Congratulations on your promotion to team leader! You have no doubt worked hard, and achieved a level of mastery of your craft. You probably know several programming languages reasonably well, and at least one particularly well.

Most team leaders are post-graduate professionals with some years of experience actively coding. You have probably been successful in writing and maintaining large software projects.

Regardless of how you arrived at this career milestone, you are to be commended that your organization saw fit to promote you. Congratulations!

But what are you expected to do? Has anyone explained your new role to you? If you are like many new team leaders, your organization may have expected you to figure that part out on your own. You are, after all, a professional problem solver. Surely figuring out your role is just another problem to solve.

You have probably heard that part of your role is to mentor others. Software engineers could often use more mentoring than they get the opportunity. Each of us is familiar with the desire to break a problem all by ourselves. There is no better way to learn the craft. But we also get stuck on hard problems, sometimes for long enough to slip schedules, or we explore a technical direction that yields massive technical debt. We could all use a good mentor when we find ourselves in trouble.

As a team leader, you will certainly need to adjust your schedule, and your own expectations. You will rarely have the freedom to code all day. You will be expected to talk to your team, and upper levels of management. You suddenly have responsibilities beyond code quality. You will need to interact more with people, as messy and complicated as they can be.

You might have a fear deep in your gut that you aren't ready to lead a team, or don't know how to handle people, or just a fear of the unknown. If you are as good as your organization thinks you are, you will take your responsibilities seriously. You will worry about your own abilities, and how to improve them.

You will already have become aware of the tendency of software projects to become much more messy than their original designs promised. You will have created beautifully simplistic data structures, only to see them become crowded with workarounds, and layered with hacks that you promised yourself would one day be refactored. Somehow, and quite suddenly, your personal responsibility for technical debt transcends your own mess. You now bear responsibility to understand and address the technical debt of your entire team. That can be an overwhelming burden.

Surely you can just code better. You can stay at work longer, and you can use your new-found skills to cover for others' newbie mistakes. You can fix it! Only your codebase isn't fixed, and you are tired. You cannot do the job of your entire team. You will know that you have gone too far down this path when you snap at a colleague in your tiredness, or give up so much of your personal time that you cease to feel like to have a life outside of work. Yielding to such temptation is common among new managers. You will find, sooner or later, that you cannot do it all.

How, then, to be a team leader? What does it mean to lead, anyway? What does leadership have to do with management? Is your new career path all about people and their problems, or will your still get to immerse yourself in software? Will you fall technically behind as you spread yourself too thin?

Where it can, this short book gives you answers. There are some. Where you will need to discover your own answers, it provides you with some new tools to help in the attempt.

Let's get the easy part out of the way: What does it mean to be a software team leader? Simply put, a software team leader must keep their team \textit{happy} and \textit{productive}. The following chapters will break those goals into manageable chunks, and describe them in more detail.

Keeping your team happy is a mnemonic for dealing with those soft, people-related, issues that tend to erode human contentedness. You may need to make a team from your team members. You will certainly need to keep that team orientation alive as times and staff change, and to balance how your team fits into the rest of your organization. Keeping people happy is mostly about defining and maintaining a group. That's what puts the team in your team leader title. You will need to become a sort of practical, empirical psychologist as you mature in your profession.

Keeping your team productive means to clear technical hurdles. That might mean writing code yourself, or helping others to do so. It might mean managing the software development methodology process used by your organization. It might even mean insulating your team from upper management so your team can get actual work done. Much of what this book discusses brings us back to people, even when the topic is technology.

\vspace{20pt}
\setlength{\fboxsep}{10pt}%
\shadowbox{
\begin{minipage}{11cm}
       \begin{center}
A software team leader must keep their team\\
\textit{happy} and \textit{productive}.
       \end{center}
       \end{minipage}
}

The next step is to realize that your fears, and your reactions to them, are not as specific to the software industry as you might believe. They are an integral part of being human. Most people feel when they are promoted that they are not fully ready for new responsibilities. Part of that feeling comes from our own emotional desires to minimize upheaval, what the keen American observer of human nature Eric Hoffer\index{Hoffer, Eric} called, ``the ordeal of change.'' It takes anyone time to adjust to new expectations, and new working conditions.

The poet T.S. Eliot\index{Eliot, T.S.} wasn't talking about software when he wrote his dark missive ``The Hollow Men''. He was pointing out the hopelessness that seeped into the European consciousness between the world wars. Every practicing software engineer, however, would recognize in these lines the frustration that occurs when a good design turns into imperfect running code:

\begin{verso}
Between the idea
And the reality
Between the motion
And the act
Falls the Shadow.
\end{verso}

Similarly, it has become common for software engineers to pass around an ancient aphorism originally attributed to the Greek physician Hippocrates\index{Hippocrates}: The life so short, the craft so long to learn. The thought is so poignant, so applicable to our everyday lives, that it has been copied, twisted, translated, and repeated for nearly two and a half millennia. The Romans loved it (as ``\textit{Ars longa, vita brevis}''), as did the ancient Jews. Geoffrey Chaucer\index{Chaucer, Geoffrey} included it in a short poem even though he left it out of the Canterbury Tales. It has appeared in modern literature up through and including rap music. Software engineers, perhaps more than most, would recognize their own struggles with the rest of Hippocrates' original sentence:

\begin{verso}
Life is short,
art long,
opportunity fleeting,
experience perilous,
decision difficult.
\end{verso}

I am not in a position to increase your lifespan, nor to change the fact that software is a never-ending opportunity to create. It is possible to provide some clues to how happy and productive software teams work. It is also possible to give you new tools to understand the dynamics of your team, both by introducing technically-oriented concepts, and people-oriented ones. Having those tools will make decision making easier.

\textbf{TODONEXT}: Everything you do will become a balancing act. Outline the various things to balance here!

\textbf{TODONEXT}: Outline the rest of the book here.


% If the chapter ends in an odd page, you may want to skip having the page
%  number in the empty page
\newpage
\thispagestyle{empty}
