%%%%%%%%%%%%%%%%%%%%%%preface.tex%%%%%%%%%%%%%%%%%%%%%%%%%%%
% 
% Lehman's Laws
% 
%%%%%%%%%%%%%%%%%%%%%%%%%%%%%%%%%%%%%%%%%%%%%%%%%%%%%%%

\begin{appendices}

\label{app:LehmansLaws}
\chapter{Lehman's Laws}

Lehman noted three types of software systems:

\begin{itemize}
\item Software systems defined entirely by their specification (S-type systems);
\item Software systems defined entirely by their procedures (P-type systems); and
\item Software systems written to perform some real-world activity, which must then evolve over time to match that activity (E-type systems).
\end{itemize}

Unsurprisingly, E-type systems form the vast bulk of software systems currently extant or planned. Lehman's Laws of software evolution apply to E-type systems.

\begin{enumerate}
\item \textbf{Continuing Change}: An E-type system must be continually adapted or it becomes progressively less satisfactory.
\item \textbf{Increasing Complexity}: As an E-type system evolves, its complexity increases unless work is done to maintain or reduce it.
\item \textbf{Self Regulation}: E-type system evolution processes are self-regulating with the distribution of product and process measures close to normal.
\item \textbf{Conservation of Organizational Stability (invariant work rate)}: The average effective global activity rate in an evolving E-type system is invariant over the product's lifetime.
\item \textbf{Conservation of Familiarity}: As an E-type system evolves, all associated with it, developers, sales personnel and users, for example, must maintain mastery of its content and behaviour to achieve satisfactory evolution. Excessive growth diminishes that mastery. Hence the average incremental growth remains invariant as the system evolves.
\item \textbf{Continuing Growth}: The functional content of an E-type system must be continually increased to maintain user satisfaction over its lifetime.
\item \textbf{Declining Quality}: The quality of an E-type system will appear to be declining unless it is rigorously maintained and adapted to operational environment changes.
\item \textbf{Feedback System}: E-type evolution processes constitute multi-level, multi-loop, multi-agent feedback systems and must be treated as such to achieve significant improvement over any reasonable base.
\end{enumerate}

Laws 1-3 were formulated in 1974, and have not changed since. Laws 4-5 were updated in 1978. Law 6 was updated in 1991. Laws 7-8 were updated in 1996. Rather amazingly, all eight laws are still considered to hold.

Lehman's Law's were defined in \cite{Lehman-1980a} and {Lehman-1980b}, and updated in \cite{Lehman-1997}. A modern modern validation may be found in \cite{Yu-2013}.

\end{appendices}

% If the chapter ends in an odd page, you may want to skip having the page
%  number in the empty page
\newpage
\thispagestyle{empty}
