%%%%%%%%%%%%%%%%%%%%%%%%%%%%%%%%%%%%%%%%%%%%%%%%%%
%
% Chapter:  Managing People
%
%%%%%%%%%%%%%%%%%%%%%%%%%%%%%%%%%%%%%%%%%%%%%%%%%%


\chapter{Managing People}

\begin{quote}
``I don't believe in art. I believe in artists.'' -- Marcel Duchamp
\end{quote}

People are not resources. They are people. They have emotions, sick cats, girlfriends, boyfriends, ambitions and bad hair days. They are also the most amazing thing in the known universe. Treat people well and they will do amazing things.  Treat them poorly and you will get what you deserve.

Software engineers are generally smarter than average. They also have a job that makes them smarter over time. The constant solving of problems physically structures and refines the organization of the cerebral cortex. Of course, there are many different ways to be smart. It shouldn't surprise you when I say that software engineers tend toward a particular category of intelligence. They should be led accordingly.

Do not expect people to respect you because you wield power over them.  Respect is earned.  If you aren't getting any, you can either work harder to earn it or find other work. You earn respect by being a leader, not a manager.  You may gain a modicum of respect by being a competent manager, but it is not the same.

People generally distrust authority, especially when it is imposed on them.  Get off to a good start.  Make utterly sure that nobody is blocked by lack of simple resources.  Give people more benefits than they must have according to their contract.  Expect them to work only the hours they are contracted for.  If you have earned their respect, the results will surprise you every time.

Good managers isolate their team from external distractions. Good managers keep their teams informed.  This is another balancing act.

\textbf{TODO:} 
\begin{itemize}
\item be careful when communicating priorities; your team might take as fixed that which is variable, such as deadlines, features, stop shop items. 
\item Work prioritization both up as well as down!
\end{itemize}


One thousand ``well dones'' can be wiped out by one ``oh, damn''. Be forgiving of others but hold yourself to higher standards.


\section{Policies}

It is impossible to administer an organization of any size without policies.  Some examples might include the requirement to perform time reporting, a security requirement to restrict access to the Internet, or a definition of what separates a sick day from a leave day. It is important to have a clear and simple set of policies appropriate to your organization.  As the organization changes (as they all do, sometimes rapidly) policies must change to stay appropriate.  Too few policies is anarchy.  Too many policies stifles an organization's morale and creativity. Policy maintenance is a balancing act.

Enforcement of policies must be consistent!  If you take into account one person's personal circumstances, then take into account other's.

An effective leadership style is to praise in public and condemn in private. Humor may be used to softly remind otherwise good staff of deviance from policies.  Repeat offenders must be counseled immediately.  Consistent offenders must have their actions documented so there is no possibility of misunderstanding. They must be given every opportunity to conform and, if they don't, punishment (up to and including firing) must occur at a logical and expected pace.  Those violating safety, security or other policies deemed unforgivable (so-called ``zero tolerance'' policies) must have been clearly told beforehand and receive immediate punishment consistent with the rest of the organization.  Failure to be consistent will invariably undermine morale. Taking individual considerations into account is yet another balancing act.

It is impossible to administer an organization without meetings.  However, meetings should always be kept short, to the point and on target.  Regularly scheduled unproductive meetings kill productivity.  Short, productive meetings enhance communication. The effective use of MWBA can and should be used to limit both the number and length of meetings.


\section{People and Personalities}

\textbf{TODO}: Summarize, describe, and suggest (careful) uses for personality modeling. Note what you can get from it, and what you can't.

\textbf{TODO}: Use material from Dr. Ingrid de Meillon.


% If the chapter ends in an odd page, you may want to skip having the page
%  number in the empty page
\newpage
\thispagestyle{empty}
