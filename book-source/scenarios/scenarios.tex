%%%%%%%%%%%%%%%%%%%%%%%%%%%%%%%%%%%%%%%%%%%%%%%%%%
%
% Chapter:  Management Scenarios
%
%%%%%%%%%%%%%%%%%%%%%%%%%%%%%%%%%%%%%%%%%%%%%%%%%%


\chapter{Management Scenarios}

\begin{quote}
``Faith is a wonderful thing, but doubt gets you an education.'' -- Wilson Mizner
\end{quote}

The following scenarios may be used as exercises for a team leader training class, or even as questions during job interviews. They are intended to make either reasonable discussion topics or written homework assignments.

All of these scenarios have occurred in my workplace during my career. You will need to think through the ramifications and decided for yourself how to keep your team happy and productive. Some possible approaches are provided, but none should be handled by simply choosing one of the provided answers. You should think about each scenario in the context of your own workplace.

It is often the case in real life that multiple approaches to a problem may be pursued in parallel. You should feel free to combine the approaches, and also to come up with your own answers.


\section{Trouble with Software}

\subsection{A Nasty Bug}

A seemingly unimportant bug is reported. Upon investigation, you determine that a major code re-architecture will be required to address the problem. What should you do?

Possible approaches:

\begin{enumerate}
\item Forget about it. If the bug is unimportant, why even consider a major change?
\item Deprioritize the bug so that it in unlikely to be discussed.
\item Think about it for a while in the hope of discovering a way around the limitation.
\item Discuss the limitation with your team, in the hope that one of them can discover a way around the limitation.
\item Inform upper management immediately, because you have discovered a limitation of your current system.
\end{enumerate}

How would your answer change if the bug was considered to be a ``stop ship'' issue for your next product release?


\subsection{Initial Setup}

A new programmer on your team is having trouble getting your product to compile on his computer. What should you do?

Possible approaches:

\begin{enumerate}
\item Don't worry about it. You expect each of your team members to overcome compilation problems on their own.
\item Leave it for a day, or a week. Eventually provide some help if absolutely necessary.
\item Allow them to work the problem for a short amount of time, then assist him.
\item Immediately assist the person by showing him how to overcome the problem.
\item Configure your team member's build environment by yourself without showing him how it was done.
\end{enumerate}


\section{Trouble with Estimation}

\subsection{The Black Art}

One of the programmers on your team consistently estimates that they can implement more than they actually can. What should you do?

Possible approaches:

\begin{enumerate}
\item Don't worry about it. Estimation doesn't really mean anything anyway. The product will get done when it is done.
\item Tell the programmer to estimate better.
\item Track the overestimations until you can spot the pattern, then apply a ``fix'' to each of the programmer's estimations.
\item Track the overestimations until you can spot the pattern, then counsel the programmer to help them estimate better.
\item Place the programmer on administrative review, and prepare to fire them if they don't improve.
\end{enumerate}


\subsection{Velocity of a Tortoise}

Your entire team is bad at estimating how long tasks will take to complete. What should you do?

Possible approaches:

\begin{enumerate}
\item Don't worry about it. Estimation doesn't really mean anything anyway. The product will get done when it is done.
\item Tell upper management that software estimation is impossible, so they should stop relying upon it.
\item Tell the team that they must get better at estimation.
\item Track the overestimations until you can spot the pattern for each programmer, then apply a ``fix'' to each of the programmer's estimations.
\item Track the overestimations until you can spot the overall pattern, then tell the team to adjust their estimates based on that information.
\item Investigate other software development methodologies to identify ways to improve your estimation.
\end{enumerate}


\section{Trouble with Facilities}

\subsection{Hot, Hot, Hot}

The air conditioning is off when you arrive at your office on a hot summer's day. Several people are already at work, and sweating profusely. You are the first senior person to arrive. What should you do?

Possible approaches:

\begin{enumerate}
\item Don't worry about it. One of the more senior managers will fix it when they get in.
\item Send a message to one of the more senior managers for their help or advice.
\item Investigate the air conditioning controls to see if you can fix it yourself.
\item Call a repair technician, and tell them to bill the company.
\item Give everyone the day off.
\end{enumerate}

How would your answer change if the only people affected are part of another team in a separate air conditioning zone to your own team?


\subsection{Hardware Failure}

You arrive at work to discover that your central revision control repository is offline. None of your team members can check in new work, but they can continue to work on their own local checkouts. What should you do?

Possible approaches:

\begin{enumerate}
\item Don't worry about it. One of the more senior managers will fix it when they get in.
\item Send a message to one of the more senior managers for their help or advice.
\item Investigate related physical machinery, logs, and software configuration to attempt to fix the problem yourself.
\item Call your systems administrator.
\item Tell your team to work locally until the problem is resolved. Assign other tasks to anyone who is blocked.
\item Give everyone the day off.
\end{enumerate}

How would your answer change if the only people affected are part of another team?

How was your approach similar to or different from your answer to the question above regarding air conditioning?


\section{Trouble with People}

This is the big one. People seem to spend their nights thinking up ways to stand out from the crowd. Not all of those ways are particularly helpful to maintaining harmony or being productive.

\subsection{A Hacker on the Team}

One of the programmers on your team is caught by you trying to crack the root password on the company's main file service. What should you do?

Possible approaches:

\begin{enumerate}
\item Don't worry about it. It is your system administrator's job to ensure the file service is unassailable.
\item Tell your systems administrator about the attempt.
\item Send a message to one of the more senior managers for their help or advice.
\item Tell everyone on your team.
\item Log into the file service if you have administrative access and stop the attempt.
\item Attempt to physically stop the perpetrator from continuing their actions.
\item Call the police and ask them to arrest the perpetrator.
\end{enumerate}

How would your answer change if you discovered that the hacker had been successful?

How would your answer change if instead of cracking the file service, the employee was instead caught reading the email of other employees?


\subsection{An Issue of Identity}

An employee arrives at work with a large pin on their shirt announcing ``I'M BISEXUAL'' in large letters followed by ``but don't tell anyone'' in nearly unreadably small letters. Some of your team members are accepting, some dismissive, and some decidedly hostile. Senior management has not yet noticed. Who will you tell, and what will you say?

Possible approaches:

\begin{enumerate}
\item Ignore it. Enforcing the organization's dress code is not your problem.
\item Don't worry about it. If it doesn't bother you, it shouldn't bother others.
\item Counsel the employee to take the pin off.
\item Counsel the team not to worry about it.
\item Tell your team that they should let their own religious or personal preferences dictate how they should treat the employee.
\item Tell your team that they should treat the employee like any any other team member, with respect and dignity.
\item Inform senior management that they should set or enforce a company-wide policy.
\end{enumerate}

How would your answer change if your were personally hostile to alternative sexualities?

How would your answer change if your organization had a policy strongly contrary to your personal opinion?


\subsection{Blissfully Unaware}

A happy and productive programmer on your team develops the annoying personal habit of loudly smacking chewing gum while concentrating. Your team members laugh about it at first, but then you begin to receive complaints. What should you do?

Possible approaches:

\begin{enumerate}
\item Don't worry about it. If it doesn't bother you, it shouldn't bother others.
\item Report the programmer to one of the more senior managers for their help or advice.
\item Move the programmer's desk away from other people.
\item Tell the disruptive employee to change their ways.
\item Place the programmer on administrative review, and prepare to fire them if they don't improve.
\end{enumerate}

How would your answer change if the disruptive employee was assigned to another team?

How would your answer change if the disruptive person was you, and you hadn't previously been conscious of it?


\subsection{A Simple Matter of Theft}

A fellow employee invites you to their apartment. Once there, you begin to recognize that the home is decorated with items belonging to your employer. Your host freely admits that they stole the materials, and presumes that you won't say anything. What should you do?

Possible approaches:

\begin{enumerate}
\item Ignore it, and pretend that you didn't notice.
\item Tell the employee that you won't say anything.
\item Tell the employee that they should return the stolen items, and that your won't say anything if they do.
\item Tell the employee that they need to tell senior management what they have done.
\item Inform senior management, and recommend that they show leniency.
\item Inform senior management, and recommend that the employee be fired.
\item Call the police and ask them to arrest the perpetrator.
\end{enumerate}

How would your answer change if you had a close personal relationship with the employee?


\section{How Bad Can It Get?}

I have experienced some decided oddities in my thirty-plus years of management. They are not common, but they can and do come out of a clear blue sky. Each of the following scenarios have also happened in my workplaces over those decades.

This is hardly a comprehensive list. The items may also serve as material for further discussion, team training, interview scenarios, or even just to point out the intrinsic humor of the human condition.

What would you do if you were faced with the following work days from hell?

\begin{enumerate}
\item An employee attempts suicide.
\item An employee is arrested while at work for domestic violence.
\item An engineering team resigns \textit{en mass} on the eve of a product launch.
\item An employee brings a goat to your office over a weekend, and leaves it there.
\item A proverbial disgruntled employee erases 80\% of the files on the company's server infrastructure overnight.
\item A fire, flood, or earthquake makes your office inaccessible.
\item Your office is robbed overnight.
\item An employee dies while at work.
\item An investor demands that your team be fired on the eve of a major holiday.
\item Your entire team arrives for work intoxicated.
\item An employee becomes addicted to an unaffordable substance, and steals important equipment from the company to support their habit.
\item Your boss leaves work with the spouse of another employee, and is subsequently chased at gunpoint by the cuckold across country for some months. Fortunately, he occasionally calls to see how you are faring.
\end{enumerate}

The last one is my personal favorite. Sometimes, all you can do is laugh.

% If the chapter ends in an odd page, you may want to skip having the page
%  number in the empty page
\newpage
\thispagestyle{empty}





